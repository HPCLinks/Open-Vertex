\documentclass[draft]{article}
\begin{document}

\begin{titlepage}
\title{XCPU Specification}
\author{Latchesar Ionkov}
\date{}
\end{titlepage}

\section{Introduction}
Xcpu is a remote process execution system that represents
execution and control services as a set of files in a hierarchical
file system. The file system can be exported and mounted  remotely
over the network.  Xcpu is slated to replace the aging B-proc cluster
management suite.

A cluster that uses Xcpu has one or more control nodes. The control
nodes represent the global view of the cluster and can be used to
execute, view and control the distributed programs. The rest of the
cluster nodes known as compute nodes and are used to run distributed
applications as guided by the control node.

Xcpu is responsible not only for the execution of the programs, but
also for their distribution to the compute nodes. It allows an
arbitrary number of files (shared libraries, configuration files)
to be pushed with the executable. In order to avoid the network
bandwidth bottleneck between the control node and the compute nodes,
Xcpu uses some of the compute nodes for the program distribution
by dedicating them as distribution nodes for the duration of the
program startup. This scheme, borrowed from B-proc, decreases
significantly the start-up time for distributed applications.

The usage of a standard file-system interface makes the system easy
to understand and operate. Furthermore, the ability to mount a
compute node over the network to the local file system is a significant
departure from the antiquated "remote execution" model in which
little or no control over the application is given to the end user.

Below is a sample session to an Xcpu compute node, which launches
a single process and displays its output on the console:
\begin{verbatim}
	$ mount -t 9p 192.168.100.101 /mnt/xcpu/1 -o port=666
	$ cd /mnt/xcpu/1
	$ ls -l
	total 0
	-r--r--r-- 1 root root 0 Jul 25 10:19 arch
	-r--r--r-- 1 root root 0 Jul 25 10:19 clone
	-rw-r--r-- 1 root root 0 Jul 25 10:19 env
	-r--r--r-- 1 root root 0 Jul 25 10:19 procs
	-r--r--r-- 1 root root 0 Jul 25 10:19 state
	$ tail -f clone &
	[1] 8237
	0tail: clone: file truncated
	$ cd 0
	$ ls -l
	total 0
	-rw-rw-rw- 1 nobody nobody 0 Jul 25 12:58 argv
	-rw-rw-rw- 1 nobody nobody 0 Jul 25 12:58 ctl
	-rw-rw-rw- 1 nobody nobody 0 Jul 25 12:58 env
	-rw-rw-rw- 1 nobody nobody 0 Jul 25 12:58 exec
	drwx------ 1 nobody nobody 0 Jul 25 12:58 fs
	-r--r--r-- 1 nobody nobody 0 Jul 25 12:58 state
	-r--r--r-- 1 nobody nobody 0 Jul 25 12:58 stderr
	-rw-rw-rw- 1 nobody nobody 0 Jul 25 12:58 stdin
	-rw-rw-rw- 1 nobody nobody 0 Jul 25 12:58 stdio
	-r--r--r-- 1 nobody nobody 0 Jul 25 12:58 stdout
	-rw-rw-rw- 1 nobody nobody 0 Jul 25 12:58 wait
	$ cp /bin/date fs
	$ echo exec date > ctl
	$ cat stdout 
	Tue Jul 25 12:59:11 MDT 2006
	$
\end{verbatim}

First, the xcpufs file system is mounted at \verb|/mnt/xcpu/1|. Reading from
the \verb|clone| file creates a new session and returns its ID. The user can
copy an arbitrary number of files to the \verb|fs| directory. The execution
of the program is done by writing \verb|exec |{\sl progname} to \verb|ctl|
file.

\section{Xcpufs}
Xcpufs is a file server that runs on all compute nodes and exports an
interface for program execution and control as a file hierarchy. The file
server uses Plan9's 9P2000 protocol. Xcpufs can be mounted on a Linux
control node using v9fs, or can be accessed directly using clients that
speak 9P2000. 

Xcpufs manages the processes it executes in sessions. In order to execute a
program, the user creates a session, copies all required files, including
the executable, sets up the argument list and the program environment and
executes the program. The user can send data to program's standard input,
read from its standard output and error, and send signals.

Only one program can be executed per session. The process of this program is
known as {\sl main session process}. That process may spawn other processes
on the compute node. Xcpufs can control (send signals, destroy) only the
main session process. 

There are two types of sessions -- normal and persistent. The {\sl normal}
session (and the subdirectory that represents it) exists as long as there is
an open file in the session directory, or the main session process is
running. The {\sl persistent} session doesn't disappear unless the user
manually {\sl wipes} it. 

Normally the program will run as the user that attached the filesystem
(field \verb|uname| in Tattach 9P2000 message). The program can be executed
as somebody else if the ownership of the session subdirectory is changed to
different user. The files in the session directory are accessible by the
owner and the members of the owners default group.

All files that are copied to a session are stored in a temporary directory
on the compute node. Before executing the program, xcpufs changes the
process' current directory to the session directory, and sets XCPUPATH
envrironment variable to it. All files in the temporary storage are
deleted when the session is destroyed.

In addition to the environment set through the env file, xcpufs adds three
more variables:

\begin{description}

\item[XCPUPATH] contains the full path to session's temprary directory
\item[XCPUID] contains the global ID of the session (same as the {\tt id} file
\item[XCPUSID] contains the local session ID (same as the session directory)

\end{description}

There are two groups of files that xcpufs exports -- top-level files that
control the xcpufs, and session files that control the individual sessions.

\subsection{Top-level files}

\begin{verbatim}
        arch
        clone
        env
	procs
	state
\end{verbatim}

\verb|Arch| is a read-only file, reading from it returns the architecture of
the compute node in a format {\sl operating-system/processor-type}.

\verb|Clone| is a read-only file. When it is opened, xcpufs creates a new
session and exports a new session directory in the file-system. It also
copies the content of the global \verb|env| file to the session one. The
open can fail if the content of the global \verb|env| file is not formatted
correctly. 

Reading from the file returns the name of the session directory. 

\verb|Env| contains the global environment to be used for all new sessions.
When a new session is created, the content of \verb|env| is copied to the
sessions \verb|env| file.

Reading from the file returns the current environment. Writing to the file
changes the environment. 

The content of the global and session environment files have the following
format:

\begin{verbatim}
	environment = *env-line
	env-line = env-name ``='' env-value LF
	env-name = ALPHA *[ALPHA | DIGIT]
	env-value = ANY
\end{verbatim}

If the \verb|env-value| contains whitespace characters (SP, TAB or LF) or
single quotes, it is enclosed in single quotes and the original single
quotes are doubled (i.e. \verb|'| becomes \verb|''|).

\verb|Procs| is a read-only file. The content of the file is a s-expression
that list all processes running on the compute node. The first subexpression
contains list of the fields returned for the processes. The list is
architecture dependent.

\verb|State| file contains the state of the node. The user can write any
string to the file.  

\subsection{Session directory}

\begin{verbatim}
        argv
        ctl
        exec
        env
	fs
        state
        stdin
        stdout
        stderr
        stdio
        wait
	id
\end{verbatim}

\subsubsection{{\tt Ctl} file}

The \verb|ctl| file is used to execute and control session's main process.

Reading from the file returns the main process pid if the process is running
and -1 otherwise.

The operations on the session are performed by writing to it. \verb|Ctl|
commands have the following format:

\begin{verbatim}
	ctl = *cmd-line
	cmd-line = command `` `` *[argument] LF
	command = ``exec'' | ``clone'' | ``wipe'' | 
		  ``signal'' | ``close'' | ``type''
	argument = ANY
\end{verbatim}

If the \verb|argument| contains whitespace characters (SP, TAB or LF) or single
quotes, it is enclosed in single quotes and the original single quotes are
doubled (i.e. \verb|'| becomes \verb|''|).

Writing to \verb|ctl| ignores the specified offset and always appends the
data to the end of the file. It is not necessary a write to contain a whole
(or single) command line. Xcpufs appends the current write data to the end
of the buffer, and executes all full command lines. The write operation
returns when all valid commands are finished.

\verb|Ctl| supports the following commands:

\begin{description}

\item[exec {\sl program-name} {\sl directory} ] Execute the program. For
backward compatibility, if program name is not specified, xcpufs executes
the program named ``xc'' from the session directory. If the {\sl directory}
is specified, xcpufs sets the current directory to that value before
executing the binary. If it is not specified, the session directory is used.

If the program-name is a relative path (i.e. doesn't start with `/'
character, the session directory path is appended in front of it.

\item[clone {\sl max-sessions address-list}] Copies the current session content
(argument list, environment and files from \verb|fs| directory) to the
specified sessions. If \verb|max-sessions| is greater than zero, and the
number of the specified sessions is bigger than \verb|max-sessions|,
\verb|clone| pushes its content to up to max-sessions and issues
\verb|clone| commands to some of them to clone themselves to the remaining
sessions from the list.

The format of the session-address is:

\begin{verbatim}
	address-list = 1*(session-address ``,'')
	session-address = [``tcp!''] node-name [``!''port] 
		          ``/'' session-id
	node-name = ANY
	port = NUMBER
	session-id = ANY
\end{verbatim}

\item [wipe] Closes the standard I/O files, if the main session process is
still alive, kills it (SIGTERM) and frees all objects used by the session.
This command is normally used to terminate persistent sessions. 

\item [signal {\sl sig}] Sends a signal to the main session process. The
signal can be specified by number, or name. The supported signal names may
depend on the node's architecture.

\item [type {\tt normal} $|$ {\tt persistent}] Changes the type of the session.

\item [close {\tt stdin} $|$ {\tt stdout} $|$ {\tt stderr}] Closes the
standard input/output/error of the main session process.

\item [id {\sl id}] Sets the session id (see the \verb|id| file). If the
job-id or the proc-id parts are ommited, they are not changed.

\end{description}

Reading from the \verb|ctl| file returns the session ID. 

Sending a signal to a session that doesn't have running process will cause
the write function to return an error.

\subsubsection{{\tt Exec} file}

The exec file is kept for backward compatibility. Writing to it creates a
file named ``exec'' in the \verb|fs| directory.

\subsection{{\tt Fs} directory}

The \verb|fs| directory points to the temporary storage created for the
session. The user can create directories and files in that directory.
Creation of special files (symbolic links, devices, named pipes) is not
allowed. 

\subsubsection{{\tt Argv} file}

Writing to the \verb|argv| file sets the program argument list. 

\verb|argv| has the following format:

\begin{verbatim}
	argument-list = 1*(argument (SP | TAB | LF))
	argument = ANY
\end{verbatim}

If the \verb|argument| contains whitespace characters (SP, TAB or LF) or single
quotes, it is enclosed in single quotes and the original single quotes are
doubled (i.e. \verb|'| becomes \verb|''|).

Reading from the \verb|argv| file returns the current content.

\subsubsection{{\tt Env} file}

When the session is created, the content of the \verb|env| file is populated
from the global \verb|env| file.

Writing to the \verb|env| file modifies the session environment.
Modifications done after the program is executed don't change its
environment.

The format of the session \verb|env| file is identical to the global one.

\subsubsection{{\tt State} file}

\verb|State| is a file that can be used both for reading and
writing. It is used by the cluster monitoring framework to mark
computational nodes' states. When Xcpu starts the state file contains
no information. If a string is written to it this string is returned
by subsequent reads.


\subsubsection{{\tt Stdin} file}

\verb|Stdin| is a write-only file. The data from the write operation is
passed to the standard input of the main session process. The write may
block if the main process doesn't consume the data.

Closing the stdin file doesn't close the standard input stream of the
main process. The file can safely be opened and closed multiple times.

\subsubsection{{\tt Stdout} file}

\verb|Stdout| is a read-only file. The read operations blocks until the main
session process writes some data to its standard output.

If the file is opened more than once, and there are blocked read
operations for these files when some data is available from the standard
output, xcpufs sends the data to every open file.

Closing the \verb|stdout| file doesn't close the standard output stream of the
main process. The file can safely be opened and closed multiple times.

\subsubsection{{\tt Stderr} file}

\verb|Stderr| is a read-only file. The read operations blocks until the main
session process writes some data to its standard error.

If the file is opened more than once, and there are blocked read
operations for these files when some data is available from the standard
output, xcpufs sends the data to every open file.

Closing the \verb|stderr| file doesn't close the standard output stream of the
main process. The file can safely be opened and closed multiple times.

\subsubsection{{\tt Stdio} file}

\verb|Stdio| file combines \verb|stdin| and \verb|stdout| functions.
Reading from \verb|stdio| is equivalent to reading from \verb|stdout|, and
writing to it is equivalent to writing to \verb|stdin|.

If the file is opened more than once, and there are blocked read
operations for these files when some data is available from the standard
output, xcpufs sends the data to every open file.

\subsubsection{{\tt Wait} file}

\verb|Wait| is a read-only file. Reading from it returns the exit code of
the main session process. The read operations will block until the process
ends.

\subsubsection{{\tt Id} file}

\verb|Id| is a read-only file that contains the user-specified id. The
format of the id is:

\begin{verbatim}
	id = job-id ``/'' proc-id
	job-id = ANY
	proc-id = NUMBER
\end{verbatim}

The job-id is a global identifier of the job, the proc-id is an id of the
process within the job. Both are set by the user via ``id'' command.

\subsection{Running Xcpufs}

Xcpufs accepts the following options:

\begin{description}

\item [-d] Optional. If set, xcpufs stays in foreground and shows all 9P2000
messages.

\item [-p port] Optional. Sets the port number that xcpufs listens on. The
default value is 666.

\item [-t tdir] Optional. Sets the directory in which xcpufs creates
the temporary session directories. The default value is /tmp.

\item [-s] Optional. If set, xcpufs runs all programs as the user that
started xcpufs.
\end{description}

\section{Other File Servers}
\subsection{Statfs}

\section{Auxillary Commands}

\subsection{Xrx}

\subsection{Xps}

\subsection{Xkill}
\subsection{Xstat}

\end{document}
